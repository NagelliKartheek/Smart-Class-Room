\chapter{Introduction}

Introductory lines...



\section{Section-1 Name}

Some text here ...

\begin{definition}\label{abc1}
Some definition....
\end{definition}

\begin{theorem}
Some theorem.......
\end{theorem}

\begin{proof}
Proof is as follows....
\end{proof}


\begin{corollary}
A corollary to the theorem is....
\end{corollary}

\begin{remark}
Some remark.......
\end{remark}


You may have to type many equations inside the text.  The equation can be typed as below.
\begin{equation}\label{eqn1}
f(x) = \frac{x^2-5x+2}{e^x - 2} = {y^5-3 \over e^x-2} %\nonumber
\end{equation}

This can be referred as (\ref{eqn1}) and so on.....

You may have to type a set of equations.  For this you may proceed as given below.
\begin{eqnarray}
f(x) &=& e^{1+2(x-a)} + \ldots   \nonumber   \\
  &=& \log(x+a) + \sin(x+y) + \cdots  \label{eqn2}
\end{eqnarray}

% Note: \nonumber will suppress the eqn number in the above.
% You can type comments like this starting with % as here.

You may have to cite the articles.  You may do so as \cite{golub} and so on.....
Note that you have already created the `bib.bib' file and included the entry with the above name. Only
then you can cite it as above.

\section{Section-2 Name}
\begin{definition}\label{abc2}
Some definition....
\end{definition}

\begin{remark}
Some remark.......
\end{remark}

\subsection{Subsection name}

\begin{theorem}
Some theorem.......
\end{theorem}

\begin{proof}
Proof is as follows.... By Definition \ref{abc1}
\end{proof}


\begin{figure}[hH]

[The figure will be displayed here.]

\caption{The correlation coefficient as a function of $\rho$}
\end{figure}


